\documentclass[11pt,]{article}
\usepackage[left=1in,top=1in,right=1in,bottom=1in]{geometry}
\newcommand*{\authorfont}{\fontfamily{phv}\selectfont}
\usepackage[]{mathpazo}


  \usepackage[T1]{fontenc}
  \usepackage[utf8]{inputenc}



\usepackage{abstract}
\renewcommand{\abstractname}{}    % clear the title
\renewcommand{\absnamepos}{empty} % originally center

\renewenvironment{abstract}
 {{%
    \setlength{\leftmargin}{0mm}
    \setlength{\rightmargin}{\leftmargin}%
  }%
  \relax}
 {\endlist}

\makeatletter
\def\@maketitle{%
  \newpage
%  \null
%  \vskip 2em%
%  \begin{center}%
  \let \footnote \thanks
    {\fontsize{18}{20}\selectfont\raggedright  \setlength{\parindent}{0pt} \@title \par}%
}
%\fi
\makeatother




\setcounter{secnumdepth}{3}

\usepackage{longtable,booktabs}

\usepackage{graphicx,grffile}
\makeatletter
\def\maxwidth{\ifdim\Gin@nat@width>\linewidth\linewidth\else\Gin@nat@width\fi}
\def\maxheight{\ifdim\Gin@nat@height>\textheight\textheight\else\Gin@nat@height\fi}
\makeatother
% Scale images if necessary, so that they will not overflow the page
% margins by default, and it is still possible to overwrite the defaults
% using explicit options in \includegraphics[width, height, ...]{}
\setkeys{Gin}{width=\maxwidth,height=\maxheight,keepaspectratio}

\title{Patrones de Asociación y Diversidad Descritos por la Familia Fabaceae
Mimosoideae en una Isla Subtropical, Cuenca del Mar Caribe.\\
\emph{Patterns of Association and Diversity Described by the Family
Fabaceae Mimosoideae on a Subtropical Island, Caribbean Sea Basin}.\\  }



\author{\Large Welifer Junior Lebron Vicente\vspace{0.05in} \newline\normalsize\emph{Estudiante de Ciencias Geográficas, Universidad Autónoma de Santo
Domingo (UASD)}  }


\date{}

\usepackage{titlesec}

\titleformat*{\section}{\normalsize\bfseries}
\titleformat*{\subsection}{\normalsize\itshape}
\titleformat*{\subsubsection}{\normalsize\itshape}
\titleformat*{\paragraph}{\normalsize\itshape}
\titleformat*{\subparagraph}{\normalsize\itshape}

\titlespacing{\section}
{0pt}{36pt}{0pt}
\titlespacing{\subsection}
{0pt}{36pt}{0pt}
\titlespacing{\subsubsection}
{0pt}{36pt}{0pt}





\newtheorem{hypothesis}{Hypothesis}
\usepackage{setspace}

\makeatletter
\@ifpackageloaded{hyperref}{}{%
\ifxetex
  \PassOptionsToPackage{hyphens}{url}\usepackage[setpagesize=false, % page size defined by xetex
              unicode=false, % unicode breaks when used with xetex
              xetex]{hyperref}
\else
  \PassOptionsToPackage{hyphens}{url}\usepackage[unicode=true]{hyperref}
\fi
}

\@ifpackageloaded{color}{
    \PassOptionsToPackage{usenames,dvipsnames}{color}
}{%
    \usepackage[usenames,dvipsnames]{color}
}
\makeatother
\hypersetup{breaklinks=true,
            bookmarks=true,
            pdfauthor={Welifer Junior Lebron Vicente (Estudiante de Ciencias Geográficas, Universidad Autónoma de Santo
Domingo (UASD))},
             pdfkeywords = {palabra clave 1, palabra clave 2},  
            pdftitle={Patrones de Asociación y Diversidad Descritos por la Familia Fabaceae
Mimosoideae en una Isla Subtropical, Cuenca del Mar Caribe.\\
\emph{Patterns of Association and Diversity Described by the Family
Fabaceae Mimosoideae on a Subtropical Island, Caribbean Sea Basin}.\\},
            colorlinks=true,
            citecolor=blue,
            urlcolor=blue,
            linkcolor=magenta,
            pdfborder={0 0 0}}
\urlstyle{same}  % don't use monospace font for urls

% set default figure placement to htbp
\makeatletter
\def\fps@figure{htbp}
\makeatother

\usepackage{pdflscape} \newcommand{\blandscape}{\begin{landscape}}
\newcommand{\elandscape}{\end{landscape}}


% add tightlist ----------
\providecommand{\tightlist}{%
\setlength{\itemsep}{0pt}\setlength{\parskip}{0pt}}

\begin{document}
	
% \pagenumbering{arabic}% resets `page` counter to 1 
%
% \maketitle

{% \usefont{T1}{pnc}{m}{n}
\setlength{\parindent}{0pt}
\thispagestyle{plain}
{\fontsize{18}{20}\selectfont\raggedright 
\maketitle  % title \par  

}

{
   \vskip 13.5pt\relax \normalsize\fontsize{11}{12} 
\textbf{\authorfont Welifer Junior Lebron Vicente} \hskip 15pt \emph{\small Estudiante de Ciencias Geográficas, Universidad Autónoma de Santo
Domingo (UASD)}   

}

}








\begin{abstract}

    \hbox{\vrule height .2pt width 39.14pc}

    \vskip 8.5pt % \small 

\noindent Mi resumen


\vskip 8.5pt \noindent \emph{Keywords}: palabra clave 1, palabra clave 2 \par

    \hbox{\vrule height .2pt width 39.14pc}



\end{abstract}


\vskip 6.5pt


\noindent  \section{Introducción}\label{introducciuxf3n}

El análisis de biodiversidad forestal viabiliza la obtención de
información sobre el comportamiento de las especies en su hábitat, los
efectos de cambios geoestacionarios, y las probables consecuencias de
actividades antrópicas en el ciclo vital de los bosques. La función de
los bosques tropicales puede ser productiva (madera, fibra, leña,
productos no maderables); ambientales (regulación del clima, reserva de
biodiversidad, conservación de suelos y agua, etc.); y social
(subsitencia de poblamientos humanos locales y su cultura) (Montagnini,
Jordan, \& others, 2005).

La isla Barro Colorado, de coordenadas {[}9º 9' 0'`N, 79º 51' 0'' W{]},
es una plataforma basáltica miocénica sobre la que descansa un bosque
tropical primario compuesto por 305 especies arboreas (Condit et al.,
1999). En el período 1981-2015 fue el emplazamiento de ocho censos
forestales realizados por el Smithsonian Tropical Research Institute,
donde la subfamilia \emph{Fabaceae Mimosoideae} representa el 5.9\% de
las especies registradas en la parcela de 50 hectareas delimitada en
1980 {[}Cita3,WebP{]}.

El registro forestal en la isla Barro colorado forma parte de una serie
de parcelas delimitadas en distintas latitudes y longitudes, pero dentro
de la zona tropical. Las parcelas poseen diferencias climiáticas
específicas con el objetivo de contabilizar, supervisar y medir
variables demográficas qeue viabilicen realizar comparaciones atendiendo
a cuestionamientos científicos, registro detalldo del comportamiento en
ecología vegetal o problemáticas resultantes de la intervención humana
en el equilibrio natural (Condit, 1998).

\begin{figure}
\centering
\includegraphics[width=0.50000\textwidth]{mapa_cuadros.png}
\caption{Área del censo forestal, Barro Colorado Island (1981-2015).}
\end{figure}

Las fabaceas se encuentran ampliamente distribuidas por la practica
totalidad de climas terrestres, concentrando su diversidad en la franja
tropical y subtropical. Están presentes en zonas árticas, litoral
costero, ambientes alpinos, bosque lluvioso, bosque estacional, sabanas,
bosque seco, desiertos áridos, pantanos y manglares. Poseen
caracteristicas especializadas que las hacen vitales para el equilibrio
ecológico y para la supervivencia del ser humano. El 88\% de las
especies de esta familia pueden formar nódulos con bacterías fijadoras
de nitrógeno (rhizobia) para fijar el N2 en la atomosfera mediante una
asociación simbiótica, fisiología rica en proteínas, etc. Asimismo, sus
semillas son empleadas para tratar celulas cancerigenas, sus
compotenentes químicos las hacen esenciales para diversos tipos de
industrias, y el grano de las leguminosas representa por si solo el 33\%
del nitrogeno necesario en la dieta de los seres humanos (Saikia, Nag,
Anurag, Chatterjee, \& Khan, 2020).

La familia \emph{Mimosoideae} dentro del clado mimosoide es una
subfamilia sumamente variable; compuesta principalmente por árboles y
arbustos de flores asimétricas cigomorfas. El clado filogenético
mimosoide es propio de climas tropicales y subtropicales, sus flores son
simétricas con petalos valvados y sus especimenes tienen un gran número
de estambres prominentes (Hasanuzzaman, Araújo, \& Gill, 2020). En BCI
se encuentran 18 de estas especies.

Atendiendo a la flexibilidad en la distribución de las fabaceas, su
importancia económica, y social; se busca entender qué factores
ambientales intervienen en la proliferación, agupamiento o decaimiento
de sus poblaciones en bosques tropicales que comparten características
con los hallados en República Dominicana, en esta ocasión tomando la
data cincuentenaria recolectada y provista por The Center for Tropical
Forest Science en BCI.

\ldots

\section{Metodología}\label{metodologuxeda}

Una vez obtenida la data censal de Barro Colorado, se crearon mapas de
distribución, agrupamiento, asociación y riqueza. Estos mapas de
carcterización de la parcela se emplearon en comparaciones con variables
presentes en el relieve, clima y edafología del lugar. Se seleccionó la
subfamilia \emph{Fabaceae Mimosoideae} dentro del total de 59 familias
registradas.

Los mapas fueron procsados mediante técnicas de ecología numérica
analizadas en R.

Fueron realizados análisis a todas

\begin{figure}
\centering
\includegraphics[width=0.50000\textwidth]{mapa_cuadros_riq_mi_familia.png}
\caption{Riqueza de especies acorde a su ubicación por cuadrante).}
\end{figure}

\ldots

\section{Resultados}\label{resultados}

La parcela de 50ha en BCI posee 3847 individuos de la familia
\emph{Fabaceae Mimosoideae} agrupados en 18 especies distribuidas de
forma aleatoria en 50 sitios de 1ha cada uno. La especie más abundante
es \emph{Inga Marginata} {[}767{]}, seguida de cerca por \emph{Inga
Umbellifera} {[}765{]}; mientras que la más escasa es \emph{Cojoba
Rufescens} {[}2{]}, seguida de \emph{Inga Oerstediana} {[}4{]}. La
abundancia especifica acorde a una organización ascendente por número de
individuos presenta una mediana de 57 individuos {[}\emph{Inga Punctata}
e \emph{Inga Laurina}{]}, siendo la mitad más pobre de especies el
equivalente a un 5.82\% {[}224{]} y la mitad más presente el 94.18\%
{[}3623{]}. La riqueza de especies por cuadrante evidencia una
distribución también desproporcional, el C26 presenta la riqueza más
débil {[}5{]} y el C30 la más fuerte {[}13{]}. No obstante, no existe
relación directa entre la riqueza y la abundancia por cuadrante, auqnue
en el caso de C26 coíncide y en algunos otros puede existir cierta
aproximación.

(ver tabla \ref{tab:abun_sp} y figura \ref{fig:abun_sp_q})

\begin{longtable}[]{@{}lr@{}}
\caption{\label{tab:abun_sp}Abundancia por especie de la familia
\emph{Fabaceae-Mimosoideae}.}\tabularnewline
\toprule
Latin & n\tabularnewline
\midrule
\endfirsthead
\toprule
Latin & n\tabularnewline
\midrule
\endhead
Inga marginata & 767\tabularnewline
Inga umbellifera & 765\tabularnewline
Inga acuminata & 606\tabularnewline
Inga nobilis & 557\tabularnewline
Inga goldmanii & 297\tabularnewline
Inga thibaudiana & 232\tabularnewline
Inga sapindoides & 197\tabularnewline
Inga pezizifera & 145\tabularnewline
Inga laurina & 57\tabularnewline
Inga punctata & 57\tabularnewline
Inga cocleensis & 54\tabularnewline
Acacia melanoceras & 48\tabularnewline
Inga spectabilis & 20\tabularnewline
Abarema macradenia & 19\tabularnewline
Enterolobium schomburgkii & 12\tabularnewline
Inga ruiziana & 8\tabularnewline
Inga oerstediana & 4\tabularnewline
Cojoba rufescens & 2\tabularnewline
\bottomrule
\end{longtable}

\begin{figure}
\centering
\includegraphics{manuscrito_files/figure-latex/unnamed-chunk-3-1.pdf}
\caption{\label{fig:abun_sp_q}Abundancia por especie por quadrat}
\end{figure}

\section{Discusión}\label{discusiuxf3n}

\section{Agradecimientos}\label{agradecimientos}

\section{Información de soporte}\label{informaciuxf3n-de-soporte}

\ldots

\section{\texorpdfstring{\emph{Script}
reproducible}{Script reproducible}}\label{script-reproducible}

\ldots

\section*{Referencias}\label{referencias}
\addcontentsline{toc}{section}{Referencias}

\hypertarget{refs}{}
\hypertarget{ref-condit1998tropical}{}
Condit, R. (1998). \emph{Tropical forest census plots: Methods and
results from barro colorado island, panama and a comparison with other
plots}. Springer Science \& Business Media.

\hypertarget{ref-condit1999dynamics}{}
Condit, R., Ashton, P. S., Manokaran, N., LaFrankie, J. V., Hubbell, S.
P., \& Foster, R. B. (1999). Dynamics of the forest communities at pasoh
and barro colorado: Comparing two 50--ha plots. \emph{Philosophical
Transactions of the Royal Society of London. Series B: Biological
Sciences}, \emph{354}(1391), 1739--1748.

\hypertarget{ref-hasanuzzaman2020plant}{}
Hasanuzzaman, M., Araújo, S., \& Gill, S. S. (2020). \emph{The plant
family fabaceae: Biology and physiological responses to environmental
stresses}. Springer Nature.

\hypertarget{ref-montagnini2005tropical}{}
Montagnini, F., Jordan, C. F., \& others. (2005). \emph{Tropical forest
ecology: The basis for conservation and management}. Springer Science \&
Business Media.

\hypertarget{ref-saikia2020tropical}{}
Saikia, P., Nag, A., Anurag, S., Chatterjee, S., \& Khan, M. L. (2020).
Tropical legumes: Status, distribution, biology and importance. In
\emph{The plant family fabaceae} (pp. 27--41). Springer.




\newpage
\singlespacing 
\end{document}
