\documentclass[11pt,]{article}
\usepackage[left=1in,top=1in,right=1in,bottom=1in]{geometry}
\newcommand*{\authorfont}{\fontfamily{phv}\selectfont}
\usepackage[]{mathpazo}


  \usepackage[T1]{fontenc}
  \usepackage[utf8]{inputenc}



\usepackage{abstract}
\renewcommand{\abstractname}{}    % clear the title
\renewcommand{\absnamepos}{empty} % originally center

\renewenvironment{abstract}
 {{%
    \setlength{\leftmargin}{0mm}
    \setlength{\rightmargin}{\leftmargin}%
  }%
  \relax}
 {\endlist}

\makeatletter
\def\@maketitle{%
  \newpage
%  \null
%  \vskip 2em%
%  \begin{center}%
  \let \footnote \thanks
    {\fontsize{18}{20}\selectfont\raggedright  \setlength{\parindent}{0pt} \@title \par}%
}
%\fi
\makeatother




\setcounter{secnumdepth}{3}

\usepackage{longtable,booktabs}

\usepackage{graphicx,grffile}
\makeatletter
\def\maxwidth{\ifdim\Gin@nat@width>\linewidth\linewidth\else\Gin@nat@width\fi}
\def\maxheight{\ifdim\Gin@nat@height>\textheight\textheight\else\Gin@nat@height\fi}
\makeatother
% Scale images if necessary, so that they will not overflow the page
% margins by default, and it is still possible to overwrite the defaults
% using explicit options in \includegraphics[width, height, ...]{}
\setkeys{Gin}{width=\maxwidth,height=\maxheight,keepaspectratio}

\title{Patrones de Ecología Numérica de Fabaceae-Mimosoideae en la Isla Barro
Colorado, Panamá.\\
\emph{Patterns of Numerical Ecology of Fabaceae-Mimosoideae in Barro
Colorado Island, Panama}.\\  }



\author{\Large Welifer Junior Lebron Vicente\vspace{0.05in} \newline\normalsize\emph{Estudiante de Ciencias Geográficas, Universidad Autónoma de Santo
Domingo (UASD)}  }


\date{}

\usepackage{titlesec}

\titleformat*{\section}{\normalsize\bfseries}
\titleformat*{\subsection}{\normalsize\itshape}
\titleformat*{\subsubsection}{\normalsize\itshape}
\titleformat*{\paragraph}{\normalsize\itshape}
\titleformat*{\subparagraph}{\normalsize\itshape}

\titlespacing{\section}
{0pt}{36pt}{0pt}
\titlespacing{\subsection}
{0pt}{36pt}{0pt}
\titlespacing{\subsubsection}
{0pt}{36pt}{0pt}





\newtheorem{hypothesis}{Hypothesis}
\usepackage{setspace}

\makeatletter
\@ifpackageloaded{hyperref}{}{%
\ifxetex
  \PassOptionsToPackage{hyphens}{url}\usepackage[setpagesize=false, % page size defined by xetex
              unicode=false, % unicode breaks when used with xetex
              xetex]{hyperref}
\else
  \PassOptionsToPackage{hyphens}{url}\usepackage[unicode=true]{hyperref}
\fi
}

\@ifpackageloaded{color}{
    \PassOptionsToPackage{usenames,dvipsnames}{color}
}{%
    \usepackage[usenames,dvipsnames]{color}
}
\makeatother
\hypersetup{breaklinks=true,
            bookmarks=true,
            pdfauthor={Welifer Junior Lebron Vicente (Estudiante de Ciencias Geográficas, Universidad Autónoma de Santo
Domingo (UASD))},
             pdfkeywords = {Ecología Numérica, BCI, Fabaceae-Mimosoideae, Lenguaje R.},  
            pdftitle={Patrones de Ecología Numérica de Fabaceae-Mimosoideae en la Isla Barro
Colorado, Panamá.\\
\emph{Patterns of Numerical Ecology of Fabaceae-Mimosoideae in Barro
Colorado Island, Panama}.\\},
            colorlinks=true,
            citecolor=blue,
            urlcolor=blue,
            linkcolor=magenta,
            pdfborder={0 0 0}}
\urlstyle{same}  % don't use monospace font for urls

% set default figure placement to htbp
\makeatletter
\def\fps@figure{htbp}
\makeatother

\usepackage{pdflscape} \newcommand{\blandscape}{\begin{landscape}}
\newcommand{\elandscape}{\end{landscape}}


% add tightlist ----------
\providecommand{\tightlist}{%
\setlength{\itemsep}{0pt}\setlength{\parskip}{0pt}}

\begin{document}
	
% \pagenumbering{arabic}% resets `page` counter to 1 
%
% \maketitle

{% \usefont{T1}{pnc}{m}{n}
\setlength{\parindent}{0pt}
\thispagestyle{plain}
{\fontsize{18}{20}\selectfont\raggedright 
\maketitle  % title \par  

}

{
   \vskip 13.5pt\relax \normalsize\fontsize{11}{12} 
\textbf{\authorfont Welifer Junior Lebron Vicente} \hskip 15pt \emph{\small Estudiante de Ciencias Geográficas, Universidad Autónoma de Santo
Domingo (UASD)}   

}

}








\begin{abstract}

    \hbox{\vrule height .2pt width 39.14pc}

    \vskip 8.5pt % \small 

\noindent Las fabáceas son plantas de amplia distribución terrestre y sus
características las hacen importantes en diversos campos de la ciencia
médica, la industria alimenticia y textil. La isla Barro Colorado posee
un registro detallado de las especies de fabáceas en 50Ha, lo suficiente
para comprender las caracteríticas ecosistémicas que condicionan el
desarrollo de esta familia en bosques tropicales. Se emplearon métodos
de análisis biogeográficos mediante lenguaje R para cuantificar su
estado, brindando información valiosa para definir las formas de
asociación, agrupamiento y diversidad de acuerdo a variables
ambientales; evidenciando que el clado filogenético mimosoide presente
no exhibe patrones asociación entre variables edáficas y geomorfológicas
específicas de manera marcada.


\vskip 8.5pt \noindent \emph{Keywords}: Ecología Numérica, BCI, Fabaceae-Mimosoideae, Lenguaje R. \par

    \hbox{\vrule height .2pt width 39.14pc}



\end{abstract}


\vskip 6.5pt


\noindent  \section{Introducción}\label{introducciuxf3n}

El análisis de biodiversidad viabiliza la obtención de información sobre
el comportamiento de las especies en su hábitat, los efectos de cambios
ambientales, las probables consecuencias de actividades antrópicas en el
ciclo vital de los bosques y en sus funciones. La importancia de los
bosques, especialmente los tropicales, radica precisamente en sus
diversas funciones productivas (e.g.~madera, fibra), ambientales
(e.g.~regulación del clima) y sociales (e.g.~subsistencia humana,
patrimonio cultural) (Montagnini, Jordan, \& others, 2005).

Las fabáceas concentran su mayor diversidad en la franja tropical y
subtropical, aunque se encuentran ampliamente distribuidas por la
práctica totalidad de climas terrestres. Están presentes en zonas
árticas, litoral costero, ambientes alpinos, bosque lluvioso, bosque
estacional, sabanas, bosque seco, desiertos áridos, pantanos y
manglares. Poseen características especializadas que las hacen vitales
para el equilibrio ecológico y para la supervivencia del ser humano.
Asimismo, el 88\% de las especies de fabáceas forman nódulos con
bacterias (rhizobia) para fijar el N\textsubscript{2} en la atmósfera
mediante asociación simbiótica, fisiología rica en proteínas, etc.;
mientras que sus semillas son empleadas para tratar células
cancerígenas, sus componentes químicos las hacen esenciales para
diversos tipos de industrias, y el grano de las leguminosas representa
el 33\% del nitrógeno necesario en la dieta del ser humano (Saikia, Nag,
Anurag, Chatterjee, \& Khan, 2020). Especificando, la subfamilia
\emph{Mimosoideae} dentro del clado filogenético mimosoide es sumamente
variable, estando compuesta principalmente por árboles y arbustos de
flores simétricas cigomorfas con pétalos valvados, a la vez que sus
especímenes tienen un gran número de estambres prominentes
(Hasanuzzaman, Araújo, \& Gill, 2020).

La isla Barro Colorado (9º 9' 0'`N, 79º 51' 0'`W), en lo adelante 'BCI',
es una plataforma basáltica miocénica sobre la que descansa un bosque
tropical primario compuesto por 305 especies arbóreas (Condit et al.,
1999). Es el emplazamiento de ocho censos forestales realizados por el
Smithsonian Tropical Research Institute entre 1981 y 2015, donde la
subfamilia \emph{Fabaceae-mimosoideae} representó el 5.9\% de las
especies registradas en la parcela de 50 hectáreas delimitada en 1980.

\begin{figure}
\centering
\includegraphics[width=0.50000\textwidth]{Map-of-Barro-Colorado-Island-BCI-Panama.png}
\caption{Isla Barro Colorado, Panamá (Baldeck et al., 2014).}
\end{figure}

El registro forestal de BCI forma parte de una serie de parcelas
permanentes delimitadas en distintas latitudes y longitudes, pero dentro
de la zona tropical. Estas parcelas poseen diferencias climáticas
específicas con el objetivo de contabilizar, supervisar y medir
variables demográficas que viabilicen realizar comparaciones atendiendo
a cuestionamientos científicos, registro detallado del comportamiento en
ecología vegetal o problemáticas resultantes de la intervención humana
en el equilibrio natural (Condit, 1998).

\begin{figure}
\centering
\includegraphics[width=0.50000\textwidth]{mapa_cuadros.png}
\caption{Área del censo forestal, Barro Colorado Island (1981-2015).}
\end{figure}

Atendiendo a la flexibilidad en la distribución de las fabáceas, su
importancia económica, y social; en el presente estudio se responde a
los niveles de asociación presentados por esta familia, la organización
de los cuadros acorde a su composición de especies, y la probable
influencia de varibales ambientales en la descripción de patrones para
estos aspectos en BCI. Además, otro objetivo es realizar una estimación
de la riqueza de la familia, asumiendo un 85\% de riqueza o más como una
`buena representación'. La determinación de estos patrones de
asociación, agrupamiento y diversidad servirán para comprender qué
caracterisricas ambientales supeditan su desarrollo. En ese sentido, la
ecología numérica es el campo de estudio que brinda las técnicas,
índices, y herramientas necesarias para obtener conclusiones a partir de
data forestal, animal, y biotopo; mientras que R es un software de
código abierto y ambiente de programación que brinda una amplia variedad
de facilidades para el manejo, creación, y visualización gráfica de
ciencia de datos, resultando ideal para este campo de estudio (Venables,
Smith, Team, \& others, 2009).

Los métodos de análisis en ecología numérica han incrementado
exponencialmente a partir de la década 1950, desarrollándose índices,
estimadores y algoritmos que permiten realizar transformaciones
cualitativas y cuantitativas para inferir y obtener precisión con
menores probabilidades de sesgo o en ausencia del mismo. La obtención de
información relativa al grado de asociación, ordenamiento, diversidad,
agrupamiento, entre otros; es debida a estas técnicas. (ver tabla
\ref {tab:met})

Table 1: Métodos de análisis en ecología numérica empleados (Moreno,
2001).\label{tab:met}

\begin{figure}
\centering
\includegraphics[width=1.00000\textwidth]{Analisis/Diversidad/Tabla_Metodos_Analisis.png}
\caption{}
\end{figure}

\ldots

\section{Metodología}\label{metodologuxeda}

La data censal de la parcela de 50Ha en BCI se obtuvo siguiendo un
meticuloso proceso de clasificación de especies, tomando en cuenta solo
especímenes con tallos de diametro mayor a 1 cm a la altura del pecho
(\emph{Data}, n.d.). Se empleó R para realizar análisis estadísticos,
gráficos, matrices, y mapas a la matriz de la familia
\emph{Fabaceae-mimosoideae} partiendo del repositorio \textbf{Scripts de
análisis BCI} (Batlle, 2020).

La creación de los utiles necesarios para los análisis fue posible
mediante el empleo de los paquetes \emph{vegan}, \emph{tydiverse}, y
\emph{sf} para extraer la familia \emph{Fabaceae-mimosoideae} de la data
censal, obtener las estadísticas, gráficos lineales y diagramas de
cajas. A la vez que, mediante \emph{mapview} se crearon, proyectaron y
almacenaron los mapas; con \emph{RColorBrewer} se para obtuvo una gama
de colores más amplia en los gráficos y mapas; y con \emph{broom} se
visualizaron las matrices de distancia. Otros paquetes empleados fueron
\emph{magrittr}, \emph{plyr}, \emph{vegetarian} ({\textbf{???}};
Appelhans, Detsch, Reudenbach, \& Woellauer, 2019; Charney \& Record,
2012; Neuwirth, 2014; Oksanen et al., 2019; Pebesma, 2018; Robinson \&
Hayes, 2019; Wickham, 2011, 2017).

Diversos estudios fueron tomados en cuenta para la clasificación de
hábitats, se emplearon 5 de las categorías definidas por Harms, Condit,
Hubbell, \& Foster (2001) partiendo del resumen y análisis de la data
compilada en 16 años, donde fueron tomados en cuenta árboles iguales o
mayores de 1 cm de diámetro a la altura del pecho. (Ver tabla
\ref{tab:hábitat})

Table 2: Clasificación de hábitats, parcela permanente BCI (Harms et
al., 2001).\label{tab:hábitat}

\begin{longtable}[]{@{}lcc@{}}
\toprule
Hábitat & Pendiente (grados) & Elevación (metros)\tabularnewline
\midrule
\endhead
Bosque adulto - Meseta baja & \textless{}7 &
\textless{}152\tabularnewline
Bosque adulto - Meseta alta & \textless{}7 & = o
\textgreater{}152\tabularnewline
Bosque adulto - Pendiente & = o \textgreater{}7 & Todas\tabularnewline
Bosque adulto - Área pantanosa & Todas & Todas\tabularnewline
Bosque joven & Todas & Todas\tabularnewline
\bottomrule
\end{longtable}

Mapas de ubicación, abundancia de individuos, riqueza de especies,
agrupamiento de parcelas, pH, nitrógeno y otras variables presentes en
el relieve, clima, y edafología del lugar se emplearon para determinar
patrones asociativos entre las especies.

La determinación de la asociación interespecífica y de sitios muéstrales
fue realizada mediante los métodos R y Q respectivamente. El modo Q se
obtuvo con la métrica de distancia euclídea o similaridad de Jaccard,
verificando la paradoja de orlóci (1978); transformando la matriz en una
de cuerdas (\emph{chord}); \emph{ji}-cuadrado; y \emph{Hellinger}. A lu
vez, el modo R se calculó con el índice de correlación de Pearson
aplicando la transformación de \emph{chi} para corregir alteraciones
producidas por outliers en los datos, convirtiendo a binaria
(presencia/ausencia) la matriz de comunidad transpuesta para calcular
distancias entre especies con la similaridad de Jaccard y el índice
\emph{rho} de Sperman. Por otro parte, el agrupamiento jerárquico fue
analizado mediante tres técnicas o criterios de enlace (siple, UPGMA,
WARD), donde fueron seleccionados los criterios enlace simple y enlace
promedio de varianza mínima tomando en cuenta métodos de regresión
lineal {[}WARD{]}, siendo la muestra dividida en tres grupos en este
último. Se descartó el método UPGMA porque los cuadros de 1Ha no
formaban grupos consistentes para análisis, siendo estos ``49-1'' al ser
dividida la muestra en dos y ``48-1-1'' al ser dividida en tres. (Ver
tabla \ref{tab:enlace})

Table 3: Clasificación de los criterios de enlace (Harms et al.,
2001).\label{tab:enlace}

\includegraphics[width=1.00000\textwidth]{Analisis/Diversidad/Tabla_Criterios_Enlace.png}
\textbar{}

Se evaluó si las varibales ambientales presentan efectos entre los
clústers de sitios agrupados según la composición de especies por el
método WARD, así como se generó la diversidad alpha aplicando la función
\emph{alpha\_div} a la matriz de comunidad, obteniendo los índices, las
entropías, equidades y ratios. En específico se crearon columnas con los
índices N0 (Renyi), entropía H (diversidad de Shannon), Hb2 (entropía de
Shannon en base =2), los números de diversidad de Hill (N1, Nb2, N2), J
(equidad de Pielou), y las dos ratios de Hill (E10 y E20).

Para obtener la estimación de riqueza de acuerdo a la diversidad de
especies en la parcela se emplearon los estimadores no paramétricos o de
``distribución libre'', entre estos están Chao1, Bootstrap, y los
Jackknife1 de primer y segundo orden.

\section{Resultados}\label{resultados}

La parcela de 50 héctareas de BCI posee 3847 individuos de la familia
\emph{Fabaceae mimosoideae} agrupados en 18 especies distribuidas de
forma aleatoria en 50 sitios de 1ha cada uno. La especie más abundante
es \emph{Inga marginata} (767 individuos), seguida de cerca por
\emph{Inga umbellifera} (765 individuos); mientras que la más escasa es
\emph{Cojoba rufescens} (2 individuos), seguida de \emph{Inga
oerstediana} (4 individuos). La abundancia de especies presenta una
mediana de 57 individuos {[}\emph{Inga punctata} e \emph{Inga
laurina}{]}, representando la mitad de especies menos abundante el
equivalente a un 5.82\% de la abundancia total (224 individuos) y la
mitad más presente el 94.18\% (3623 individuos). La riqueza de especies
por cuadro de 1 Ha en BCI, en lo adelante `Cxx' para cuadros
específicos, evidencia una distribución también desigual, el C26
presenta la riqueza más débil (5 especies) y el C30 la más fuerte (13
especies). (ver tabla \ref{tab:abun_sp} y figura \ref{fig:abun_sp_q})

\begin{longtable}[]{@{}lr@{}}
\caption{\label{tab:abun_sp}Abundancia por especie de la familia
\emph{Fabaceae-Mimosoideae}.}\tabularnewline
\toprule
Latin & n\tabularnewline
\midrule
\endfirsthead
\toprule
Latin & n\tabularnewline
\midrule
\endhead
Inga marginata & 767\tabularnewline
Inga umbellifera & 765\tabularnewline
Inga acuminata & 606\tabularnewline
Inga nobilis & 557\tabularnewline
Inga goldmanii & 297\tabularnewline
Inga thibaudiana & 232\tabularnewline
Inga sapindoides & 197\tabularnewline
Inga pezizifera & 145\tabularnewline
Inga laurina & 57\tabularnewline
Inga punctata & 57\tabularnewline
Inga cocleensis & 54\tabularnewline
Acacia melanoceras & 48\tabularnewline
Inga spectabilis & 20\tabularnewline
Abarema macradenia & 19\tabularnewline
Enterolobium schomburgkii & 12\tabularnewline
Inga ruiziana & 8\tabularnewline
Inga oerstediana & 4\tabularnewline
Cojoba rufescens & 2\tabularnewline
\bottomrule
\end{longtable}

\begin{figure}
\centering
\includegraphics{manuscrito_files/figure-latex/unnamed-chunk-3-1.pdf}
\caption{\label{fig:abun_sp_q}Abundancia por especie por quadrat}
\end{figure}

La forma del terreno predominante es la vertiente, siendo esta
característica la más destacada en el 88\% de los cuadros; en tanto que
el 12\% restante describe una geomorfología donde predomina el relieve
llano. La parcela de 50 Ha está compuesta en un 52\% por bosque adulto
en meseta baja distribuido ampliamente, un 24\% por bosque adulto en
pendiente con presencia marcada en las parcelas {[}C41-C45{]}, un 16\%
por bosque adulto en meseta alta concentrado en las parcelas {[}C32-C34;
C37-C40{]}, mientras que el 8\% restante es hábitat pantanoso y bosque
joven en forma equitativa. No parece existir una relación directa entre
la morfología del espacio y un determinado hábitat, exceptuando el
bosque joven que se encuentra tanto en llanura como en vertiente de
forma considerable. En general, se percibe un leve aumento de la riqueza
específica a medida que aumenta la abundancia de individuos; destacando
la abundante riqueza de especies en el hábitat pantanoso, su pobreza en
el bosque adulto en zona alta, la abundancia marcada el bosque adulto en
zona baja y su riqueza equilibrada. (Ver figura \ref{fig:leal}).

\begin{figure}
\centering
\includegraphics[width=1.00000\textwidth]{Analisis/Diversidad/Graf_regre_lineal_aed_2.png}
\caption{Distribución de sitios en función de su riqueza, abundancia, y
hábitat dominante.\label{fig:leal}}
\end{figure}

El grado de asociación según la similaridad de Jaccard acorde a la
abundancia de especies por cuadro con matriz transformada
\emph{Hellinger} expone la existencia de clústers sumamente semejantes
limitados por los sitios C28-C33; C33-C40; C24-C2; C8-C10. Estos sitios
comparten un 75\% o más de las especies de acuerdo con la matriz
ordenada en función de la relación de proximidad (ver mapa de calor
superior \ref{fig:heat}). Por otra parte, la presencia de variables
edáficas como minerales y otros elementos está asociada a la abundancia
de especies en el clúster formado por los sitios C1-C9 de la matriz
ordenada en función del id de lugar, pero además de esta relación no hay
indicios de una dependencia entre estas variables que evidencie
asociación alguna (ver mapa de calor inferior \ref{fig:heat}). De igual
forma, no existe una relación de asociación entre las variables mixtas
``hábitat, quebrada, heterogeneidad ambiental'' y la distribución de
especies por cuadros. (Ver figura \textbackslash{}ref\{fig:heat).

\begin{figure}
\centering
\includegraphics[width=1.00000\textwidth]{Analisis/Imagenes manuscrito/Heat_maps_Q.png}
\caption{Mapas asociación de especies por abundancia en los sitios de
muestreo (superior) y por asociación de variables ambientales y
abundancia (inferior). El color azul fuerte representa ``lejanía'' y el
rosado fuerte ``proximidad''.\label{fig:heat}}
\end{figure}

La asociación interespecífica en función de la abundancia, como se
observa en el mapa de calor \ref{fig:heat1}, evidencia un grado de
asociación muy amplio {[}75\%-100\%{]} en el clúster delimitado por
\emph{Inga lauriana} e \emph{Inga margitana}, así como uno considerable
en el macroclúster comprendido entre \emph{Inga pezizifera} e \emph{Inga
spectabilis}. Es descrito un patrón similar por las especies en la
matriz binaria (presencia/ausencia), donde se forma un clúster entre
\emph{Inga cocleensis} e \emph{Inga laurina} que describe bastante
coexistencia entre las especies {[}75\%-100\%{]}, y un macroclúster
entre \emph{Inga pezizifera} e \emph{Inga spectabilis} con cercanía
considerable. En ese sentido, la matriz de correlación entre las
variables ``abundancia, riqueza, y composición del suelo'' , realizada
mediante el índice \emph{rho} de sperman, corrobora lo observado en los
mapas de calor anteriores acerca del escaso grado de asosiación entre
dichas variables. Tampoco se observa relación entre las dos primeras y
las variables relativas a la geomorfología del lugar. (Ver figura
\ref{fig:sper})

\begin{figure}
\centering
\includegraphics[width=1.00000\textwidth]{Analisis/Imagenes manuscrito/Heat_Map_R.png}
\caption{Mapas de calor, asociación entre especies en función de su
abundancia (superior)y en función de su abundancia con matriz binaria
``presencia/ausencia'' (inferior).\label{fig:heat1}}
\end{figure}

\begin{figure}
\centering
\includegraphics[width=1.00000\textwidth]{Analisis/Imagenes manuscrito/Seperman_R.png}
\caption{Matrices de Sperman, asociación de especies en función de las
variables edáficas (superior) y asociación de especies en función de las
variables geomorfológicas (inferior) .\label{fig:sper}}
\end{figure}

Acorde al ordenamiento por enlace simple, los cuadros 32, 50 y 43 se
separan bastante de los demás, lo cual se corresponde con la abundancia
y riqueza de especies en estos sitios comparada con los demás, siendo en
C32 116 {[}tercera más alta{]} y 8 {[}tercil más pobre{]}
respectivamente, por ejemplo. También se observan dos grupos definidos
formados entre los sitios C22-C37 y C28-C7. (Ver \label{fig:denS})

\begin{figure}
\centering
\includegraphics[width=1.00000\textwidth]{DenSimple.png}
\caption{Dendrograma a partir del criterio enlace
simple.\label{fig:denS}}
\end{figure}

En el dendrograma de WARD se observa la presencia de tres grupos
definidos entre los sitios C3-C7, C46-C34 y C27-C23; formados por 11, 18
y 21 sitios respectivamente. También pueden definirse cinco subgrupos
menores más próximos entre sí: C3-C7 {[}sitios con abundancia y riqueza
promedio{]}, C46-C22, C27-C48, C50-C23 {[}citios con las mayores
abundancias y riquezas, o al menos una de estas en forma de outlier{]}
(ver figura\ref{fig:denW}).Continuando con los tres primeros grupos WARD
mencionados, en los diagramas de cajas para las variables ambientales se
observa distancia que se acrecienta según se cambia de grupo, en orden
ascendente, en las medianas respecto a las variables Boro, Hierro,
Geomorfología en llanura, Potasio, Nitrógeno Mínimo, pH y zinc. Esta
tendencia es también descrita en menor medida por las variables Calcio,
Cobre, Heterogeneidad ambiental y Magnesio. En contraste, este patrón se
invierte en las variables Aluminio y Geomorfología en vaguada. Por otra
parte, el segundo grupo se decanta por la variable Fósforo, presentando
números inferiores en Abundancia y Riqueza de especies, mientras que el
primer y tercer grupo son bastante homogéneos en ese sentido.

\begin{figure}
\centering
\includegraphics[width=1.00000\textwidth]{Inter_Ward_5Clusters_Dendrogram.png}
\caption{Dendrograma a partir del criterio enlace promedio
(WARD).\label{fig:denW}}
\end{figure}

\begin{figure}
\centering
\includegraphics[width=1.00000\textwidth]{WARD_Plots_Variables_Ambientales_Clusters.png}
\caption{Diagramas de cajas en función de las variables ambientales para
los grupos WARD.\label{fig:boxW}}
\end{figure}

En la matriz de correlación para determinar equidad se observan niveles
de correlación bastante bajos para todas las variables ambientales,
siendo las variables Boro, Fósforo, Nitrógeno y pH las que presentan
niveles significativos respecto a los números de Hill y los índices de
Shannon. Por otra parte, la variable ``Riqueza global'' presenta niveles
determinantes para los índices anteriores, promediando un 73\% de
correlación colectivamente.

\begin{figure}
\centering
\includegraphics[width=1.00000\textwidth]{Analisis/Diversidad/Indices_Env_Diversidad_Alpha_2.png}
\caption{Matriz de diversidad alpha: correlación entre índices,
entropías, equidades, y ratios con las variables
ambientales.\label{fig:divA}}
\end{figure}

La riqueza de especies respecto a la diversidad de estas en la parcela
se determinó en porcentajes aproximados a la riqueza real. Por otro
lado, la diversidad beta describe una curva ascendente según aumenta el
número de ordenamiento en Hill, llegando a superar el N0. Presentando un
grado de dominancia mayor y una dependencia de la abundancia en la
diversidad beta. Las especies que más contribuyen a la diversidad son
\emph{Inga acumiata}, \emph{Inga marginata}, \emph{Inga Pezizifera} e
\emph{Inga thibaudiana}; donde las dos primeras son de las especies más
abundantes y por si solas representan el 35.7\% de los individuos en la
comunidad de las fabáceas. Por otra parte, los sitios que más
contribuyen a la diversidad beta son C32 y C50. (Ver tabla
\ref{tab:est})

Table 4: Estimadores de riqueza de especies. \label{tab:est}

\begin{longtable}[]{@{}llllll@{}}
\toprule
Estimadores & Chao & Jackknife1 & Jackknife2 & Bootstrap &
Real\tabularnewline
\midrule
\endhead
& 18 (100\%) & 18 (100\%) & 17.06 (105.5\%) & 18.19 (98.95\%) &
18\tabularnewline
\bottomrule
\end{longtable}

\section{Discusión}\label{discusiuxf3n}

La familia \emph{Fabaceae mimosoideae} en BCI presenta niveles de
asociación (coexistencia) sobre el 75\% entre las especies \emph{Inga
laurina}, \emph{Inga margitana}, \emph{Inga thitaudiana}, \emph{Inga
sapindoides}, \emph{Inga umbellifera}, \emph{Inga goldimanii} e
\emph{Inga nobilis}. Estas especies son las más abundantes de esta
familia en BCI, exceptuando por \emph{Inga laurina} que tomando en
cuenta sus apenas 57 individuos se encuentra ampliamente distribuida.
Ninguna variable ambiental resultó determinante en la existencia de
especies en los cuadros, aunque existen relaciones significativas entre
las variables edáficas Al {[}grupo 1{]}, K {[}grupo 2{]}, pH y Zn
{[}grupo 3{]}.

La composición de las fabáceas es bastante uniforme tomando en cuenta
que a través del criterio de enlace promedio UPGMA no se pudo dividir
las muestras de forma efectiva en dos grupos {[}resultando en 49 y 1{]}.
Sin embargo, mediante el método promedio de la varianza mínima se pudo
separar efectivamente la muestra en tres grupos de cuadros, los cuales
presentaron medianas bastante aproximadas entre sí para la mayoría de
variables ambientales.

La diversidad de las especies no está afectada de forma significativa
por alguna variable, siendo la equidad poco significativa para todas las
variables ambientales, exceptuando por la abundancia global que es la
determinante de la diversidad en la muestra. En ese sentido, la riqueza
las fabáceas respecto a la diversidad es bastante alta, teniendo
porcentajes mayores al 98\% en todos los índices calculados. Cuatro
especies contribuyen significativamente a la dversidad: \emph{Inga
marginata}, \emph{Inga acumiata}, \emph{Inga thibaudiana} e \emph{Inga
Pezizifera}, todas ellas pertenecientes a la mitad más abundante de las
especies.

\section{Agradecimientos}\label{agradecimientos}

Este trabajo fue posible gracias al Ph.D José Ramón Martinez Battle,
quien supervisó todo el proceso, brindó la asesoría y fuentes
bibliográficas que sirvieron de base para la elaboración del estudio.

\section{Información de soporte}\label{informaciuxf3n-de-soporte}

\ldots

\section{\texorpdfstring{\emph{Script}
reproducible}{Script reproducible}}\label{script-reproducible}

\ldots

\hypertarget{refs}{}
\hypertarget{ref-mapview}{}
Appelhans, T., Detsch, F., Reudenbach, C., \& Woellauer, S. (2019).
\emph{Mapview: Interactive viewing of spatial data in r}. Retrieved from
\url{https://CRAN.R-project.org/package=mapview}

\hypertarget{ref-inproceedings}{}
Baldeck, C., Asner, G., Martin, R., Anderson, C., Knapp, D., Kellner,
J., \& Wright, S. J. (2014). Operational tree species mapping in a
diverse tropical forest with airborne imaging spectroscopy. \emph{PloS
one}, \emph{10}. \url{https://doi.org/10.1371/journal.pone.0118403}

\hypertarget{ref-jose_ramon_martinez_batlle_2020_4402362}{}
Batlle, J. R. M. (2020). biogeografia-master/scripts-de-analisis-BCI:
Long coding sessions (Version v0.0.0.9000).
\url{https://doi.org/10.5281/zenodo.4402362}

\hypertarget{ref-vegetarian}{}
Charney, N., \& Record, S. (2012). \emph{Vegetarian: Jost diversity
measures for community data}. Retrieved from
\url{https://CRAN.R-project.org/package=vegetarian}

\hypertarget{ref-condit1998tropical}{}
Condit, R. (1998). \emph{Tropical forest census plots: Methods and
results from barro colorado island, panama and a comparison with other
plots}. Springer Science \& Business Media.

\hypertarget{ref-condit1999dynamics}{}
Condit, R., Ashton, P. S., Manokaran, N., LaFrankie, J. V., Hubbell, S.
P., \& Foster, R. B. (1999). Dynamics of the forest communities at pasoh
and barro colorado: Comparing two 50--ha plots. \emph{Philosophical
Transactions of the Royal Society of London. Series B: Biological
Sciences}, \emph{354}(1391), 1739--1748.

\hypertarget{ref-BCIdata}{}
\emph{Data: Forest census plot on barro colorado island}. (n.d.).
\url{http://ctfs.si.edu/webatlas/datasets/bci/}.

\hypertarget{ref-harms2001habitat}{}
Harms, K. E., Condit, R., Hubbell, S. P., \& Foster, R. B. (2001).
Habitat associations of trees and shrubs in a 50-ha neotropical forest
plot. \emph{Journal of Ecology}, \emph{89}(6), 947--959.

\hypertarget{ref-hasanuzzaman2020plant}{}
Hasanuzzaman, M., Araújo, S., \& Gill, S. S. (2020). \emph{The plant
family fabaceae: Biology and physiological responses to environmental
stresses}. Springer Nature.

\hypertarget{ref-montagnini2005tropical}{}
Montagnini, F., Jordan, C. F., \& others. (2005). \emph{Tropical forest
ecology: The basis for conservation and management}. Springer Science \&
Business Media.

\hypertarget{ref-book}{}
Moreno, C. (2001). \emph{Métodos para medir la biodiversidad} (Vol. 1).

\hypertarget{ref-RcolorBrewer}{}
Neuwirth, E. (2014). \emph{RColorBrewer: ColorBrewer palettes}.
Retrieved from \url{https://CRAN.R-project.org/package=RColorBrewer}

\hypertarget{ref-vegan1}{}
Oksanen, J., Blanchet, F. G., Friendly, M., Kindt, R., Legendre, P.,
McGlinn, D., \ldots{} Wagner, H. (2019). \emph{Vegan: Community ecology
package}. Retrieved from \url{https://CRAN.R-project.org/package=vegan}

\hypertarget{ref-sf}{}
Pebesma, E. (2018). Simple Features for R: Standardized Support for
Spatial Vector Data. \emph{The R Journal}, \emph{10}(1), 439--446.
\url{https://doi.org/10.32614/RJ-2018-009}

\hypertarget{ref-broom}{}
Robinson, D., \& Hayes, A. (2019). \emph{Broom: Convert statistical
analysis objects into tidy tibbles}. Retrieved from
\url{https://CRAN.R-project.org/package=broom}

\hypertarget{ref-saikia2020tropical}{}
Saikia, P., Nag, A., Anurag, S., Chatterjee, S., \& Khan, M. L. (2020).
Tropical legumes: Status, distribution, biology and importance. In
\emph{The plant family fabaceae} (pp. 27--41). Springer.

\hypertarget{ref-venables2009introduction}{}
Venables, W. N., Smith, D. M., Team, R. D. C., \& others. (2009).
\emph{An introduction to r}. Citeseer.

\hypertarget{ref-plyr}{}
Wickham, H. (2011). The split-apply-combine strategy for data analysis.
\emph{Journal of Statistical Software}, \emph{40}(1), 1--29. Retrieved
from \url{http://www.jstatsoft.org/v40/i01/}

\hypertarget{ref-tydiverse}{}
Wickham, H. (2017). \emph{Tidyverse: Easily install and load the
'tidyverse'}. Retrieved from
\url{https://CRAN.R-project.org/package=tidyverse}




\newpage
\singlespacing 
\end{document}
